\documentclass[12pt]{article}
\usepackage[T1]{fontenc}
\usepackage[utf8]{inputenc}
\usepackage{mathptmx}
\usepackage{enumitem}
\usepackage[paperheight=11in,paperwidth=8.5in,margin=0.75in]{geometry}

\usepackage{fancyhdr}
\pagestyle{fancy}
\rhead{Shrinu Kushagra (Ref. no: 396158785)}

\begin{document}
\linespread{0.901}

\noindent\textbf{Part I – Contributions to research and development}

\begin{itemize}[noitemsep]
\item  \textbf{Kushagra, Shrinu}, Ashtiani, Hassan and Shai Ben-David. "Clustering with same-cluster queries." In Advances in Neural Information Processing Systems, 2016 (To appear).

Oral presentation (46 accepted out of 2500 submissions). One of the best conferences in Machine Learning (International conference). Work done during doctoral degree.

\item  \textbf{Kushagra, Shrinu}, Samadi, Samira and Shai Ben-David. "Finding meaningful cluster structure amidst background noise." In International Conference on Algorithmic Learning Theory, 2016 (To appear).

Oral presentation. International conference. Work done during doctoral degree.
 
\item \textbf{Kushagra, Shrinu}, and Shai Ben-David. "Information Preserving Dimensionality Reduction." In International Conference on Algorithmic Learning Theory, pp. 239-253. Springer International Publishing, 2015.

Oral presentation. International conference. Work done during the completion of master's degree.

\item \textbf{Kushagra, Shrinu}, Alejandro Lopez-Ortiz, J. Ian Munro, and Aurick Qiao. "Multi-pivot quicksort: Theory and experiments." In Proceedings of the Meeting on Algorithm Engineering \& Expermiments, pp. 47-60. Society for Industrial and Applied Mathematics, 2014.

Oral presentation. International conference. Work done during the completion of master's degree.

\item Ji, Xian$g^*$, \textbf{Shrinu Kushagra}, and Jeff Orchard. "Sensory updates to combat path-integration drift." In Canadian Conference on Artificial Intelligence, pp. 263-270. Springer Berlin Heidelberg, 2013.

Poster presentation. International conference. Work done during the completion of master's degree.
\end{itemize}

\noindent\textbf{Part II – Most significant contributions to research and development}\\\\
Clustering with same-cluster queries
\begin{itemize}[noitemsep]
\item I was one of the main contributors of this project both in terms of research contribution and technical writing. I wrote the first draft of the manuscript which was later edited by my coauthors Shai Ben-David (phd advisor) and Mr. Hassan Ashtiani (phd student at Univeristy of Waterloo). 
\item The project had two major components. The first one involved designing an algorithm and proving that it works when our input satisfies certain conditions. This result was obtained with active collaboration with my coauthor Mr. Ashtiani. The second part was much more challenging. This involved proving that if the conditions are not satisfied then no algorithm can find the solution in a reasonable amount of time. This direction, including the proof, was mainly investigated by me.
\item This work will appear in Neural Information Processing Systems, NIPS 2016. It has been accepted for an oral presentation. NIPS is one of the best conferences in Machine Learning. This year around 2500 submissions were made. Out of those only 46 were accepted for an oral presentation. The submission was reviewed by six anonymous reviewers.
\item We prove that the task of clustering is computationally expensive. However, with the help of weak human supervision, we can transform this computationally expensive task into an easy one. The result is highly encouraging and surprising and very few such results are known in the computer science literature. One of the reviewers even remarked that "Novelty/originality of the work, theory, algorithm and applications are ground-breaking and potentially seminal". 
\end{itemize}

\vspace{0.15in}\noindent Finding meaningful cluster structure amidst background noise
\begin{itemize}[noitemsep]
\item I was one of the main contributors of this project both in terms of research contribution and technical writing. I wrote the first draft of the manuscript which was later edited by my coauthors Shai Ben-David and Ms. Samira Samadi (currently phd student at Georgia Institute of Technology). 
\item This work will appear in Algorithmic Learning Theory, ALT 2016. It has been accepted for an oral presentation. ALT is one of the top-tier conferences in Theoretical Machine Learning and the submission was reviewed by three anonymous reviewers.
\item In this work, we introduced a novel notion to capture noise in realistic datasets. We introduce efficient algorithms that discover and cluster every subset of the data with meaningful structure. We further show that when either the notions of structure or the noise requirements are relaxed, no such results are possible. I was the principal investigator on all the research directions of this project. 
\item This work lays the theoretical foundation for clustering under reasonable assumptions on data. 
\end{itemize}

\vspace{0.15in}\noindent Multi-pivot quicksort: Theory and experiments
\begin{itemize}[noitemsep]
\item I and Mr. Aucrick Qiao (now phd student at Carnegie Mellon Univeristy) were the main contributors on this project both in terms of research contribution and technical writing. The first draft was jointly written by us. This was later edited by my other coauthors Ian Munro and Alex Ortiz. 
\item This work appeared in Algorithm Engineering and Experiments, ALENEX 2014 as an oral presentation. Quicksort is a classical algorithm for sorting a list of numbers. Up until about a decade ago, it was thought that the classic quicksort algorithm is superior to any multi-pivot scheme. However, in 2009 a dual-pivot algorithm was proposed which outperformed the standard algorithm. We proposed a three-pivot quicksort variant. We proved, theoretically and experimentally that our algorithm has better performance than other schemes. While me and my coauthor aurick qiao actively collaborated to obtain the experimental evaluations, the theoretical results were proved by me.
\item Since its publication, this work has been cited $20$ times. The algorithm has also been introduced into first year computer science courses at Princeton university. The algorithm and methods developed in this work have also been discussed in graduate seminar courses at Standford University. 
\end{itemize}

\vspace{0.2in}\noindent\textbf{Part III – Applicant's statement}\\\\
\textbf{Research Experience}

\vspace{0.1in}\noindent Describe the scientific or engineering abilities that you have gained through your past research experience, including special projects, honours thesis and co-op reports. If you have relevant work experience, discuss the relevance of that experience to your proposed field of study/research and any benefits you gained from it. Do not repeat any information you provided in Part II. \\

\noindent\textbf{Relevant activities}

\vspace{0.1in}\noindent Describe your professional and extracurricular activities that most demonstrate your communication, interpersonal, and leadership skills. Examples of these include: 
\begin{itemize}[noitemsep]
\item oral presentations
\item mentoring
\item teaching
\item project management
\item chairing committees
\item organizing conferences or meetings
\item supervisory experience
\item elected positions held and volunteer work 
\end{itemize}
\end{document}