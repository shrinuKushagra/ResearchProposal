\documentclass[12pt]{article}
\usepackage[T1]{fontenc}
\usepackage[utf8]{inputenc}
\usepackage{mathptmx}

\usepackage[paperheight=11in,paperwidth=8.5in,margin=0.75in]{geometry}

\usepackage{fancyhdr}
\pagestyle{fancy}
\rhead{Shrinu Kushagra (PIN: 504441)}

\begin{document}
\linespread{0.901}
In July 2016, Uber started its self-driving car pickup services in Pittsburg. The guardian reports that by $2020$ most of our cars will be driven by computers and humans will have a permanent backseat. Our phones are now becoming smarter. Apple's siri, for example, can converse like a human-being. Amazon's echo can answers questions, read audio-books, report traffic, provide sports scores among various other things. It has been reported that by 2025, professional jobs like accountants, lawyers, doctors could be done by or be heavily-assisted by machines. Some small startups have already started building programs that could automate all your accounting needs. These are just some of the many ways in which `Artificial Intelligence' is transforming our lives.\\

Computers `learn' in two ways. The first class of programs (or algorithms) are called \textit{supervised} methods where the algorithm learns to identify patterns (like sound or text of the alphabet 'A') with the help of a human being.  The second class of methods are \textit{unsupervised} where the algorithms learns to detect and distinguish patterns by themselves. Lot of the recent success in AI like autonomous cars etc. can be attributed to advances in unsupervised learning algorithms. \\

However, research in unsupervised learning has still not been able to explain the success of some of these algorithms. Under what situations should one algorithm be preferred over the other? What conditions are needed for a certain algorithm succeed? Even more importantly, under what conditions would the given algorithm fail? Current research has little to no answer to these questions. These questions become very critical, like in the domain of self-driving cars, where human lives are at stake. In April 2016, EU even approved a law giving its citizens a \textit{right to explanation} of algorithmic decisions. My research aims to answer the above questions and fill this knowledge gap. I am applying for this scholarship  to support my research on {Foundations of clustering}, a popular class of unsupervised learning algorithms. 
\end{document}